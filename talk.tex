\documentclass[11pt]{beamer}

\usecolortheme[light,accent=blue]{solarized}
\setbeamercovered{transparent=0}
\setbeamertemplate{navigation symbols}{} % remove navigation symbols
\setbeamertemplate{section in toc}[sections numbered] % numbered sections
% \setbeamertemplate{footline}[page number]
\setbeamertemplate{footline}{
  \hfill%
  \usebeamercolor[fg]{page number in head/foot}%
  \usebeamerfont{page number in head/foot}%
  \setbeamertemplate{page number in head/foot}[framenumber]%
  \usebeamertemplate*{page number in head/foot}\kern1em\vskip2pt%
}
\setbeamerfont{page number in head/foot}{size=\small}
\setbeamerfont{note page}{size=\scriptsize}
\addtobeamertemplate{note page}{\setbeamerfont{itemize/enumerate subbody}{size=\tiny}}{}
\setbeamertemplate{bibliography item}{\insertbiblabel}

\usepackage[english]{babel}
% \usepackage{times}
% \usepackage{esvect}\renewcommand{\vec}[1]{\vv{#1}}
% \usepackage{setspace}\setstretch{1.5}
\usepackage{amsmath,amssymb}
\usepackage{physics}
\usepackage{slashed}
\usepackage{tikz}
\usepackage{pgfplots}\pgfplotsset{compat=1.17}
\usepackage{listings}
\usepackage{siunitx}
\usepackage[backend=biber,natbib=true,style=numeric,sorting=none]{biblatex}
% \usepackage{multimedia}

\addbibresource{talk.bib}

%%%%%%%%%%%%%%%%%%%%%%%%%%%%%%%%%%%%%%%%%%%%%%%%%%

\usetikzlibrary{calc,decorations.markings,decorations.pathmorphing,positioning,shapes,scopes}

\tikzset{
  >=latex,
  ->-/.style={postaction={decorate},decoration={%
      markings,mark=at position #1 with {\arrow{>}}%
    }%
  },%
  ->-/.default=.5,
  -<-/.style={postaction={decorate},decoration={%
      markings,mark=at position #1 with {\arrowreversed{>}}%
    }%
  },%
    -<-/.default=.5,
  % use option [visible on=<+->] to uncover parts of a tikzpicture
  invisible/.style={opacity=0},
  visible on/.style={alt=#1{}{invisible}},
  alt/.code args={<#1>#2#3}{%
    \alt<#1>{\pgfkeysalso{#2}}{\pgfkeysalso{#3}} % \pgfkeysalso doesn't change the path
  },
  photon/.style={decorate, decoration={snake, segment length=3mm, amplitude=0.8mm}},
}

% arrows on the field lines
\tikzstyle directed=[postaction={decorate,decoration={markings,
  mark=at position .2 with {\arrowreversed[scale=1.5]{>}},
  mark=at position .8 with {\arrowreversed[scale=1.5]{>}}}}]

% field lines
\tikzstyle fLines=[thick,directed]

\tikzset{xcenter around/.style 2 args={execute at end picture={%
  \useasboundingbox let \p0 = (current bounding box.south west), \p1 = (current bounding box.north east),
                        \p2 = (#1), \p3 = (#2)
                    in
        ({min(\x2 + \x3 - \x1,\x0)},\y0) rectangle ({max(\x3 + \x2 - \x0,\x1)},\y1);
}}}

\lstset{ 
  backgroundcolor=\color{white},   % choose the background color; you must add \usepackage{color} or \usepackage{xcolor}; should come as last argument
  basicstyle=\footnotesize\ttfamily,        % the size of the fonts that are used for the code
  breaklines=true,                 % sets automatic line breaking
  stepnumber=1,
  prebreak=\textbackslash,
  commentstyle=\color{gray},       % comment style
  % deletekeywords={...},            % if you want to delete keywords from the given language
  % escapeinside={\%*}{*)},          % if you want to add LaTeX within your code
  % extendedchars=true,              % lets you use non-ASCII characters; for 8-bits encodings only, does not work with UTF-8
  frame=single,	                   % adds a frame around the code
  keepspaces=true,                 % keeps spaces in text, useful for keeping indentation of code (possibly needs columns=flexible)
  keywordstyle=\color{blue},       % keyword style
  morekeywords={wget},             % if you want to add more keywords to the set
  numbers=left,                    % where to put the line-numbers; possible values are (none, left, right)
  numbersep=5pt,                   % how far the line-numbers are from the code
  numberstyle=\tiny\color{gray},   % the style that is used for the line-numbers
  rulecolor=\color{gray},         % if not set, the frame-color may be changed on line-breaks within not-black text (e.g. comments (green here))
  showspaces=false,                % show spaces everywhere adding particular underscores; it overrides 'showstringspaces'
  showstringspaces=false,          % underline spaces within strings only
  showtabs=false,                  % show tabs within strings adding particular underscores
  % stringstyle=\color{red},         % string literal style
  tabsize=2,	                   % sets default tabsize to 2 spaces
}

\newcommand{\CXX}{\texttt{C++}}
\newcommand{\FORTRAN}{\texttt{FORTRAN}}
\newcommand{\FlexibleSUSY}{\texttt{FlexibleSUSY}}
\newcommand{\HiggsTools}{\texttt{HiggsTools}}
\newcommand{\ii}{\text{i}}
\newcommand{\hc}{\text{h.c.}}
\newcommand{\Lagr}{\mathcal{L}}
\newcommand{\Lagrin}{\mathcal{L}_{\text{int}}}
\newcommand{\micrOMEGAs}{\texttt{micrOMEGAs}}
\newcommand{\SARAH}{\texttt{SARAH}}
\newcommand{\SPheno}{\texttt{SPheno}}

\newcommand{\SMtable}{%
    \begin{tikzpicture}[node distance = 2.5em, auto]
      \node[quark] (u) {$u$};
      \node[quark, below of=u] (d) {$d$};
      \node[quark, right of=u] (c) {$c$};
      \node[quark, below of=c] (s) {$s$};
      \node[quark, right of=c] (t) {$t$};
      \node[quark, below of=t] (b) {$b$};
      \node[lepton, below of=d] (ne) {$\nu_e$};
      \node[lepton, below of=ne] (e) {$e$};
      \node[lepton, right of=ne] (nm) {$\nu_\mu$};
      \node[lepton, below of=nm] (m) {$\mu$};
      \node[lepton, right of=nm] (nt) {$\nu_\tau$};
      \node[lepton, below of=nt] (ta) {$\tau$};
      \node[gauge, right of=t] (gamma) {$\gamma$};
      \node[gauge, below of=gamma] (g) {$g$};
      \node[gauge, below of=g] (Z) {$Z$};
      \node[gauge, below of=Z] (W) {$W$};
      \node[scalar, right of=W] (H) {$h$};
      \node[rotate=90] (quarks)  at ($(u)!0.5!(d)+(-1,0)$)  {quarks};
      \node[rotate=90] (leptons) at ($(ne)!0.5!(e)+(-1,0)$) {leptons};
      \node[below of=H] (higgs) {Higgs};
      \node[above of=gamma, align=center] (gauge) {gauge\\[-0.5em] bosons};
    \end{tikzpicture}
}

%%%%%%%%%%%%%%%%%%%%%%%%%%%%%%%%%%%%%%%%%%%%%%%%%%

\tikzstyle{block} = [rectangle, draw, text width=7em, text centered, minimum height=2em]

\tikzstyle{quark}     = [rectangle, black, draw, fill=yellow, minimum width=2em, text centered, minimum height=2em]
\tikzstyle{lepton}    = [rectangle, black, draw, fill=red!50, minimum width=2em, text centered, minimum height=2em]
\tikzstyle{gauge}     = [circle   , black, draw, fill=green , minimum size=2em, inner sep=0pt, text centered]
\tikzstyle{scalar}    = [diamond  , black, draw, fill=blue!40, minimum width=2.3em, text centered, minimum height=2.3em, inner sep=0pt]
\tikzstyle{goldstone} = [diamond  , black, draw, dashed, fill=blue!30, minimum width=2.3em, text centered, minimum height=2.3em, inner sep=0pt]
\tikzstyle{squark}    = [diamond  , black, draw, fill=yellow, minimum width=2.3em, text centered, minimum height=2.3em, inner sep=0pt]
\tikzstyle{slepton}   = [diamond  , black, draw, fill=red!50, minimum width=2.3em, text centered, minimum height=2.3em, inner sep=0pt]
\tikzstyle{gaugino}   = [rectangle, black, draw, fill=green , minimum size=2em, inner sep=0pt, text centered]
\tikzstyle{higgsino}  = [rectangle, black, draw, fill=blue!40  , minimum width=2em, text centered, minimum height=2em]
\tikzstyle{inert}     = [diamond  , black, draw, fill=teal!80, minimum width=2.3em, text centered, minimum height=2.3em, inner sep=0pt]
\tikzstyle{inertino}  = [rectangle, black, draw, fill=teal!80, minimum width=2em, text centered, minimum height=2em]
\tikzstyle{phantom}   = [rectangle, black, minimum width=2em, text centered, minimum height=2em]

%%%%%%%%%%%%%%%%%%%%%%%%%%%%%%%%%%%%%%%%%%%%%%%%%%

\newcommand{\Exp}{\text{Exp}}
\newcommand{\SM}{\text{SM}}

\definecolor{red}{rgb}{1.0,0.2,0.2}
\definecolor{blue}{rgb}{0,0.7,1.0}
\definecolor{green}{rgb}{0,1.0,0.5}

%%%%%%%%%%%%%%%%%%%%%%%%%%%%%%%%%%%%%%%%%%%%%%%%%%

\title{Tools for HEP}
\subtitle{From $\Lagr$ to Observables}

\author[Voigt]{Alexander Voigt}
\institute[RWTH Aachen]{RWTH Aachen}
\date{Workshop on ``Selected topics on future directions in particle physics'', Warsaw 2024}

\begin{document}

%%%%%%%%%%%%%%%%%%%%%%%%%%%%%%%%%%%%%%%%%%%%%%%%%%

\begin{frame}
  \titlepage
\end{frame}

%%%%%%%%%%%%%%%%%%%%%%%%%%%%%%%%%%%%%%%%%%%%%%%%%%

\begin{frame}{Table of Contents}
  \tableofcontents
\end{frame}

%%%%%%%%%%%%%%%%%%%%%%%%%%%%%%%%%%%%%%%%%%%%%%%%%%

\section{Overview}

\begin{frame}{Overview}
  \begin{center}
    \begin{tikzpicture}[node distance=1.5em, auto]
      \node[block] (L) { Lagrangian $\Lagr$ };
      \node[block, below=of L] (E) { expressions for vertices, $M_f$, $\Sigma_f$, $\beta_i$, \ldots };
      \node[block, left=of E] (S) { \SPheno };
      \node[block, below=of S] (SO) { values for $m_f$, $\Gamma$, $a_\mu$, \ldots };
      \node[block, right=of E] (F) { \FlexibleSUSY };
      \node[block, below=of F] (FO) { values for $m_f$, $\Gamma$, $a_\mu$, \ldots };
      \node[block, below=4em of E] (O) { \micrOMEGAs, \HiggsTools, \ldots };
      \node[block, below=of O] (O2) { values for $\Omega h^2$, $\chi^2$, \ldots };
      \draw[->] (L) -- node[right] { \SARAH } (E);
      \draw[->] (E) -- (S);
      \draw[->] (E) -- (F);
      \draw[->] (S) -- (SO);
      \draw[->] (F) -- (FO);
      \draw[->] (SO) |- (O);
      \draw[->] (FO) |- (O);
      \draw[->] (E) -- (O);
      \draw[->] (O) -- (O2);
    \end{tikzpicture}
  \end{center}
\end{frame}

\note{
  \begin{itemize}
  \item Start with picture of the galaxy cluster ``Abell 1689'' from the Hubble telescope
  \item look closely: you'll see a gravity lens effect
  \item So, there must be some massive object(s) between the galaxy
    cluster and us.
  \item If one counts the number of visible objects (stars), one finds
    that (assuming ART is correct), the cumulative mass of the stars
    is not enough to explain this gravity lens effect.
  \item So, there must be some invisible massive matter between the
    galaxy cluster and us. This is what astronomers and cosmologists
    call ``Dark Matter''.
  \end{itemize}
}

%%%%%%%%%%%%%%%%%%%%%%%%%%%%%%%%%%%%%%%%%%%%%%%%%%

\section{Setup}
\subsection{Mathematica}

\begin{frame}[fragile]{\insertsection\ -- Mathematica}
  \begin{enumerate}
  \item Get a free licence for WolframEngine:
    \begin{center}
      \url{https://www.wolfram.com/engine/free-license/}
    \end{center}
  \item Download an install WolframEngine 14.1 from:
    \begin{center}
      \url{https://www.wolfram.com/engine/}
    \end{center}
  \item Ensure the Wolfram Language Kernel can be called from the
    command line by running \texttt{math}:
    \begin{lstlisting}[language=sh]
$PATH=$PATH:/usr/local/Wolfram/WolframEngine/14.1/Executables/\end{lstlisting}
  \item Test it:
    \begin{lstlisting}
math -run "Print[Hello World]; Quit[]"\end{lstlisting}
  \end{enumerate}
\end{frame}

%%%%%%%%%%%%%%%%%%%%%%%%%%%%%%%%%%%%%%%%%%%%%%%%%%

\subsection{SARAH}

\begin{frame}[fragile]{\insertsection\ -- \SARAH\ 4.15.2}
  \textbf{Goal:} The following command should work:
  \begin{lstlisting}
math -run "<< SARAH\`"\end{lstlisting}
  It should print (or similar):
  \begin{lstlisting}[basicstyle=\scriptsize\ttfamily]
Wolfram Language 14.1.0 Engine for Linux x86 (64-bit)
Copyright 1988-2023 Wolfram Research, Inc.

SARAH 4.15.2
by Florian Staub, Mark Goodsell and Werner Porod, 2020
contributions by M. Gabelmann, K. Nickel\end{lstlisting}
\end{frame}

%%%%%%%%%%%%%%%%%%%%%%%%%%%%%%%%%%%%%%%%%%%%%%%%%%

\begin{frame}[fragile]{\insertsection\ -- \SARAH\ 4.15.2}
  \textbf{Method 1:} automatic download and installation:
  \begin{lstlisting}[language=sh,basicstyle=\scriptsize\ttfamily]
wget https://raw.githubusercontent.com/FlexibleSUSY/FlexibleSUSY/development/install-sarah
chmod +x install-sarah
./install-sarah --flavour=WolframEngine\end{lstlisting}
  \textbf{Method 2:} manual download:
  \begin{lstlisting}[language=sh,basicstyle=\scriptsize\ttfamily]
cd ~/.WolframEngine/Applications
wget https://sarah.hepforge.org/downloads/SARAH-4.15.2.tar.gz
tar -xf SARAH-4.15.2.tar.gz
ln -s SARAH-4.15.2 SARAH\end{lstlisting}
  and installation:
  \begin{lstlisting}[language=sh,basicstyle=\scriptsize\ttfamily]
cd ~/.WolframEngine/Kernel
echo 'AppendTo[$Path, FileNameJoin[{$HomeDirectory, ".WolframEngine", "Applications", "SARAH"}]];' >> init.m
\end{lstlisting}%$
\end{frame}

%%%%%%%%%%%%%%%%%%%%%%%%%%%%%%%%%%%%%%%%%%%%%%%%%%

\subsection{Compilers and Libraries}

\begin{frame}[fragile]{\insertsection\ -- compilers and libraries}
  \textbf{Requirements:} \CXX\ compiler, \FORTRAN\ compiler, BOOST
  library, Eigen3 library, GNU Scientific Library

  \bigskip

  Installation on \textbf{Ubuntu}:
  \begin{lstlisting}[language=sh]
sudo apt install gcc g++ gfortran libboost-dev libeigen3-dev libgsl-dev dpkg-dev\end{lstlisting}
  %
  Installation on \textbf{MacOS}:
  \begin{lstlisting}[language=sh]
brew install gcc boost eigen gsl\end{lstlisting}
\end{frame}

%%%%%%%%%%%%%%%%%%%%%%%%%%%%%%%%%%%%%%%%%%%%%%%%%%

\subsection{FlexibleSUSY}

\begin{frame}[fragile]{\insertsection\ -- \FlexibleSUSY}
  \begin{lstlisting}[language=sh]
wget https://flexiblesusy.hepforge.org/downloads/FlexibleSUSY-2.8.0.tar.gz
tar -xf FlexibleSUSY-2.8.0.tar.gz
cd FlexibleSUSY-2.8.0\end{lstlisting}%$
\end{frame}

%%%%%%%%%%%%%%%%%%%%%%%%%%%%%%%%%%%%%%%%%%%%%%%%%%

\section{Standard Model}

\begin{frame}{\insertsection\ (SM)}
  Interaction Lagrangian in terms of Weyl 2-Spinors:
  \begin{align*}
    \Lagrin
    &= \qty[-Y^d_{ij} H_a^* d_iq_{aj} - Y^e_{ij} H_a^* e_i\ell_{aj} - Y^u_{ij} u_iq_{aj}H_a + \hc] \\
    &\phantom{=}{} -\mu^2 H_a^* H_a - \frac{1}{2} (H_a^* H_a)^2
  \end{align*}
\end{frame}

%%%%%%%%%%%%%%%%%%%%%%%%%%%%%%%%%%%%%%%%%%%%%%%%%%

\begin{frame}[fragile]{\insertsection}
  Create and build the SM spectrum generator:
  \begin{lstlisting}
./createmodel -f --name=SM
./configure --with-models=SM
make -j4\end{lstlisting}
  Run it:
  \begin{lstlisting}
./models/SM/run_SM.x --slha-input-file=models/SM/LesHouches.in.SM\end{lstlisting}

%   \begin{lstlisting}
% wget https://github.com/FlexibleSUSY/FlexibleSUSY/releases/download/v2.8.0/SM.tar.gz
% tar xf SM.tar.gz
% ./configure --with-models=SM --disable-meta
% make -j4\end{lstlisting}
\end{frame}

%%%%%%%%%%%%%%%%%%%%%%%%%%%%%%%%%%%%%%%%%%%%%%%%%%

\section{Standard Model + S}

\begin{frame}{\insertsection}
\end{frame}

%%%%%%%%%%%%%%%%%%%%%%%%%%%%%%%%%%%%%%%%%%%%%%%%%%

\begin{frame}[allowframebreaks]{References}
  \printbibliography
\end{frame}

\end{document}
