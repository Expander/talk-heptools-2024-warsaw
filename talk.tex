\documentclass[12pt]{beamer}

\usecolortheme[light,accent=blue]{solarized}
\setbeamercovered{transparent=0}
\setbeamertemplate{navigation symbols}{} % remove navigation symbols
% \setbeamertemplate{footline}[page number]
\setbeamertemplate{footline}{
  \hfill%
  \usebeamercolor[fg]{page number in head/foot}%
  \usebeamerfont{page number in head/foot}%
  \setbeamertemplate{page number in head/foot}[framenumber]%
  \usebeamertemplate*{page number in head/foot}\kern1em\vskip2pt%
}
\setbeamerfont{page number in head/foot}{size=\small}
\setbeamerfont{note page}{size=\scriptsize}
\addtobeamertemplate{note page}{\setbeamerfont{itemize/enumerate subbody}{size=\tiny}}{}
\setbeamertemplate{bibliography item}{\insertbiblabel}

\usepackage[english]{babel}
% \usepackage{times}
% \usepackage{esvect}\renewcommand{\vec}[1]{\vv{#1}}
% \usepackage{setspace}\setstretch{1.5}
\usepackage{tikz}
\usepackage{pgfplots}\pgfplotsset{compat=1.17}
\usepackage{siunitx}
\usepackage[backend=biber,natbib=true,style=numeric,sorting=none]{biblatex}
% \usepackage{multimedia}

\addbibresource{talk.bib}

%%%%%%%%%%%%%%%%%%%%%%%%%%%%%%%%%%%%%%%%%%%%%%%%%%

\usetikzlibrary{calc,decorations.markings,decorations.pathmorphing,positioning,shapes,scopes}

\tikzset{
  >=latex,
  ->-/.style={postaction={decorate},decoration={%
      markings,mark=at position #1 with {\arrow{>}}%
    }%
  },%
  ->-/.default=.5,
  -<-/.style={postaction={decorate},decoration={%
      markings,mark=at position #1 with {\arrowreversed{>}}%
    }%
  },%
    -<-/.default=.5,
  % use option [visible on=<+->] to uncover parts of a tikzpicture
  invisible/.style={opacity=0},
  visible on/.style={alt=#1{}{invisible}},
  alt/.code args={<#1>#2#3}{%
    \alt<#1>{\pgfkeysalso{#2}}{\pgfkeysalso{#3}} % \pgfkeysalso doesn't change the path
  },
  photon/.style={decorate, decoration={snake, segment length=3mm, amplitude=0.8mm}},
}

% arrows on the field lines
\tikzstyle directed=[postaction={decorate,decoration={markings,
  mark=at position .2 with {\arrowreversed[scale=1.5]{>}},
  mark=at position .8 with {\arrowreversed[scale=1.5]{>}}}}]

% field lines
\tikzstyle fLines=[thick,directed]

\tikzset{xcenter around/.style 2 args={execute at end picture={%
  \useasboundingbox let \p0 = (current bounding box.south west), \p1 = (current bounding box.north east),
                        \p2 = (#1), \p3 = (#2)
                    in
        ({min(\x2 + \x3 - \x1,\x0)},\y0) rectangle ({max(\x3 + \x2 - \x0,\x1)},\y1);
}}}

\newcommand{\CXX}{\texttt{C++}}
\newcommand{\FORTRAN}{\texttt{FORTRAN}}
\newcommand{\FlexibleSUSY}{\texttt{FlexibleSUSY}}
\newcommand{\HiggsTools}{\texttt{HiggsTools}}
\newcommand{\Lagr}{\mathcal{L}}
\newcommand{\micrOMEGAs}{\texttt{micrOMEGAs}}
\newcommand{\SARAH}{\texttt{SARAH}}
\newcommand{\SPheno}{\texttt{SPheno}}

\newcommand{\SMtable}{%
    \begin{tikzpicture}[node distance = 2.5em, auto]
      \node[quark] (u) {$u$};
      \node[quark, below of=u] (d) {$d$};
      \node[quark, right of=u] (c) {$c$};
      \node[quark, below of=c] (s) {$s$};
      \node[quark, right of=c] (t) {$t$};
      \node[quark, below of=t] (b) {$b$};
      \node[lepton, below of=d] (ne) {$\nu_e$};
      \node[lepton, below of=ne] (e) {$e$};
      \node[lepton, right of=ne] (nm) {$\nu_\mu$};
      \node[lepton, below of=nm] (m) {$\mu$};
      \node[lepton, right of=nm] (nt) {$\nu_\tau$};
      \node[lepton, below of=nt] (ta) {$\tau$};
      \node[gauge, right of=t] (gamma) {$\gamma$};
      \node[gauge, below of=gamma] (g) {$g$};
      \node[gauge, below of=g] (Z) {$Z$};
      \node[gauge, below of=Z] (W) {$W$};
      \node[scalar, right of=W] (H) {$h$};
      \node[rotate=90] (quarks)  at ($(u)!0.5!(d)+(-1,0)$)  {quarks};
      \node[rotate=90] (leptons) at ($(ne)!0.5!(e)+(-1,0)$) {leptons};
      \node[below of=H] (higgs) {Higgs};
      \node[above of=gamma, align=center] (gauge) {gauge\\[-0.5em] bosons};
    \end{tikzpicture}
}

%%%%%%%%%%%%%%%%%%%%%%%%%%%%%%%%%%%%%%%%%%%%%%%%%%

\tikzstyle{block} = [rectangle, draw, text width=7em, text centered, minimum height=2em]

\tikzstyle{quark}     = [rectangle, black, draw, fill=yellow, minimum width=2em, text centered, minimum height=2em]
\tikzstyle{lepton}    = [rectangle, black, draw, fill=red!50, minimum width=2em, text centered, minimum height=2em]
\tikzstyle{gauge}     = [circle   , black, draw, fill=green , minimum size=2em, inner sep=0pt, text centered]
\tikzstyle{scalar}    = [diamond  , black, draw, fill=blue!40, minimum width=2.3em, text centered, minimum height=2.3em, inner sep=0pt]
\tikzstyle{goldstone} = [diamond  , black, draw, dashed, fill=blue!30, minimum width=2.3em, text centered, minimum height=2.3em, inner sep=0pt]
\tikzstyle{squark}    = [diamond  , black, draw, fill=yellow, minimum width=2.3em, text centered, minimum height=2.3em, inner sep=0pt]
\tikzstyle{slepton}   = [diamond  , black, draw, fill=red!50, minimum width=2.3em, text centered, minimum height=2.3em, inner sep=0pt]
\tikzstyle{gaugino}   = [rectangle, black, draw, fill=green , minimum size=2em, inner sep=0pt, text centered]
\tikzstyle{higgsino}  = [rectangle, black, draw, fill=blue!40  , minimum width=2em, text centered, minimum height=2em]
\tikzstyle{inert}     = [diamond  , black, draw, fill=teal!80, minimum width=2.3em, text centered, minimum height=2.3em, inner sep=0pt]
\tikzstyle{inertino}  = [rectangle, black, draw, fill=teal!80, minimum width=2em, text centered, minimum height=2em]
\tikzstyle{phantom}   = [rectangle, black, minimum width=2em, text centered, minimum height=2em]

%%%%%%%%%%%%%%%%%%%%%%%%%%%%%%%%%%%%%%%%%%%%%%%%%%

\newcommand{\Exp}{\text{Exp}}
\newcommand{\SM}{\text{SM}}

\definecolor{red}{rgb}{1.0,0.2,0.2}
\definecolor{blue}{rgb}{0,0.7,1.0}
\definecolor{green}{rgb}{0,1.0,0.5}

%%%%%%%%%%%%%%%%%%%%%%%%%%%%%%%%%%%%%%%%%%%%%%%%%%

\title{Tools for HEP}
\subtitle{From $\Lagr$ to Observables}

\author[Voigt]{Alexander Voigt}
\institute[RWTH Aachen]{RWTH Aachen}
\date{Workshop on ``Selected topics on future directions in particle physics'', Warsaw 2024}

\begin{document}

%%%%%%%%%%%%%%%%%%%%%%%%%%%%%%%%%%%%%%%%%%%%%%%%%%

\begin{frame}
  \titlepage
\end{frame}

%%%%%%%%%%%%%%%%%%%%%%%%%%%%%%%%%%%%%%%%%%%%%%%%%%

\begin{frame}{Table of Contents}
  \tableofcontents
\end{frame}

%%%%%%%%%%%%%%%%%%%%%%%%%%%%%%%%%%%%%%%%%%%%%%%%%%

\section{Overview}

\begin{frame}{}
  \begin{center}
    \begin{tikzpicture}[node distance=1.5em, auto]
      \node[block] (L) { Lagrangian $\Lagr$ };
      \node[block, below=of L] (E) { expressions for vertices, $M_f$, $\Sigma_f$, $\beta_i$, \ldots };
      \node[block, left=of E] (S) { \SPheno };
      \node[block, below=of S] (SO) { values for $m_f$, $\Gamma$, $a_\mu$, \ldots };
      \node[block, right=of E] (F) { \FlexibleSUSY };
      \node[block, below=of F] (FO) { values for $m_f$, $\Gamma$, $a_\mu$, \ldots };
      \node[block, below=2cm of E] (O) { \micrOMEGAs, \HiggsTools, \ldots };
      \node[block, below=of O] (O2) { values for $\Omega h^2$, $\chi^2$, \ldots };
      \draw[->] (L) -- node[right] { \SARAH } (E);
      \draw[->] (E) -- (S);
      \draw[->] (E) -- (F);
      \draw[->] (S) -- (SO);
      \draw[->] (F) -- (FO);
      \draw[->] (SO) |- (O);
      \draw[->] (FO) |- (O);
      \draw[->] (E) -- (O);
      \draw[->] (O) -- (O2);
    \end{tikzpicture}
  \end{center}
\end{frame}

\note{
  \begin{itemize}
  \item Start with picture of the galaxy cluster ``Abell 1689'' from the Hubble telescope
  \item look closely: you'll see a gravity lens effect
  \item So, there must be some massive object(s) between the galaxy
    cluster and us.
  \item If one counts the number of visible objects (stars), one finds
    that (assuming ART is correct), the cumulative mass of the stars
    is not enough to explain this gravity lens effect.
  \item So, there must be some invisible massive matter between the
    galaxy cluster and us. This is what astronomers and cosmologists
    call ``Dark Matter''.
  \end{itemize}
}

%%%%%%%%%%%%%%%%%%%%%%%%%%%%%%%%%%%%%%%%%%%%%%%%%%

\section{The Standard Model}

\begin{frame}{\insertsection}
  \begin{center}
    \SMtable
  \end{center}
\end{frame}

\note{
  \begin{itemize}
  \item To calculate the quantum corrections we have to take into
    account all known particles.
  \item All known particles and their interactions are described by
    the SM.
  \item So, to predict $g_e$ one must take into account all known
    particles from the SM which make up the ``sea'' of virtual
    particles.
  \item The known particles described by the SM are: Electron, Muon,
    Tau, Quarks, Gauge bosons and Higgs boson.
  \end{itemize}
}

%%%%%%%%%%%%%%%%%%%%%%%%%%%%%%%%%%%%%%%%%%%%%%%%%%

\begin{frame}[allowframebreaks]{References}
  \printbibliography
\end{frame}

\end{document}
